%%%%%%%%%%%%%%%%%%%%%%%%%%%%%%%%%%%%%%%%%%%%%%%%%%%%%%%%%%%%%%%%%%%%
%%%           Vorlage für eine Ausarbeitung an der DHBW          %%%
%%%                                                              %%%
%%%      Bereiche die bearbeitet werden müssen werden durch      %%%
%%%      einen solchen Kommentarblock eingeleitet und enden      %%%
%%%      mit der nächsten Trennlinie.                            %%%
%%%                                                              %%%
%%%      In dieser Datei müssen folgende Bereiche bearbeitet     %%%
%%%      werden:                                                 %%%
%%%      - Angaben zur Arbeit                                    %%%
%%%      - EIGENE KAPITEL EINFÜGEN                               %%%
%%%                                                              %%%
%%%      Benötigte Seiten und Verzeichnisse können unter         %%%
%%%      "Einführung und Verzeichnisse" ein- bzw. auskommentiert %%%
%%%      werden.                                                 %%%
%%%                                                              %%%
%%%%%%%%%%%%%%%%%%%%%%%%%%%%%%%%%%%%%%%%%%%%%%%%%%%%%%%%%%%%%%%%%%%%

\documentclass[a4paper,12pt]{article}
\usepackage[left=2.5cm,right=2.5cm,top=2.5cm,bottom=2.5cm,includehead]{geometry}      % Einstellungen der Seitenränder
\usepackage[english, ngerman]{babel}                                                  % deutsche Silbentrennung
\usepackage[utf8]{inputenc}                                                           % Umlaute
\usepackage[T1]{fontenc}													                                    % Umlaute auch richtig ausgeben
\usepackage{newtxtext,newtxmath}                                                      % Font = Times New Roman
\usepackage{hyperref}
\usepackage[nottoc]{tocbibind}
\usepackage{fancyhdr}
\usepackage{setspace}
\usepackage[backend=bibtex, citestyle=authoryear, bibstyle=authoryear]{biblatex}      % Bibliothek für Zitate
\usepackage{csquotes}                                                                 % Zusatzpacket für Zitate
\usepackage{amsmath}                                                                  % Zurücksetzen der Tabellen- und Abbildungsnummerierung je Sektion
\usepackage[labelfont=bf,aboveskip=1mm]{caption}                                      % Bild- und Tabellenunterschrift (fett)
\usepackage[bottom,multiple,hang,marginal]{footmisc}                                  % Fußnoten [Ausrichtung unten, Trennung durch Seperator bei mehreren Fußnoten]
\usepackage{graphicx}  
\graphicspath{{./images/}}                                                            % Grafiken
\usepackage[dvipsnames]{xcolor}                                                       % Farbige Buchstaben
\usepackage{wrapfig}                                                                  % Bilder in Text integrieren
\usepackage{enumitem}                                                                 % Befehl setlist (Zeilenabstand für itemize Umgebung auf 1 setzen)
\usepackage{listings}                                                                 % Quelltexte
\definecolor{commentgreen}{RGB}{87,166,74}                                            % Kommentar-Farbe für Quellcode
\lstset{numbers=left, numberstyle=\tiny, numbersep=8pt, frame=single, framexleftmargin=15pt, breaklines=true, commentstyle=\color{commentgreen}}
\usepackage{tabularx}                                                                 % Tabellen
\usepackage{multirow}                                                                 % Mehrzeilige Tabelleneinträge
\usepackage[addtotoc]{abstract}                                                       % Abstract
\usepackage[nohyperlinks, printonlyused, withpage]{acronym}                           % Abkürzungen
\usepackage{dirtree}                                                                  % Ordnerstruktur (z.B. für Anhang)
\usepackage{float}
\usepackage{pdfpages}

%%%%%%%%%%%%%%%%%%%%%%%%%%%%%%%%%%%%%%%%%%%%%%%%%%%%%%%%%%%%%%%%%%%%
%%%                      Angaben zur Arbeit                      %%%
%%%%%%%%%%%%%%%%%%%%%%%%%%%%%%%%%%%%%%%%%%%%%%%%%%%%%%%%%%%%%%%%%%%%
\def\vFirmenlogoPfad{}                  %% relativer Pfad Bsp.: images/Firmenlogo.png
\def\vDHBWLogoPfad{images/DHBW_logo.jpg}                          %% relativer Pfad Bsp.: images/DHBW_logo.jpg
\def\vUnterschrift{}                    %% Pfad zu Bild mit Unterschrift (für digitale Abgabe) Bsp.: images/Unterschrift.png

\def\vTitel{Softwarequalität}                           %% 
\def\vUntertitel{}                      %% 
\def\vArbeitstyp{Hausarbeit}                      %% Projektarbeit/Seminararbeit/Bachelorarbeit
\def\vArbeitsbezeichnung{}              %% T1000/T2000/T3000

\def\vLB{Lukas Braun}
\def\vJB{Johannes Brandenburger}
\def\vHS{Henry Schuler}
\def\vPP{Phillipp Patzelt}


\def\vAutor{\vJB, \vLB, \vPP, \vHS}                           %% Vorname Nachname
\def\vMatrikelnummer{}                  %% 7-stellige Zahl
\def\vKursKuerzel{TIT20}                     %% Bsp.: TIT20
\def\vPhasenbezeichnung{Theoriephasen}               %% Praxisphase/Theoriephase
\def\vStudienJahr{dritte}                     %% erste/zweite/dritte
\def\vDHBWStandort{Ravensburg}                    %% Bsp.: Ravensburg
\def\vDHBWCampus{Friedrichshafen}                      %% Bsp.: Friedrichshafen
\def\vFakultaet{Technik}                       %% Technik/Wirtschaft
\def\vStudiengang{Informatik}                     %% Informationstechnik/...
\def\vKurs{TIT20}                     %% IT/...

\def\vBearbeitungsort{Friedrichshafen}                 %%                       %% 
\def\vBetreuer{Benjamin Jung}                        %% Vorname Nachname

\def\vAbgabedatum{\today}               %% DD. MONTH YYYY
\def\vBearbeitungszeitraum{01.10.2022 - 06.01.2023}            %% DD.MM.YYYY - DD.MM.YYYY
%TODO Datum anpassen

%%%%%%%%%%%%%%%%%%%%%%%%% Eigene Kommandos %%%%%%%%%%%%%%%%%%%%%%%%%
% Definition von \gqq{}: Text in Anführungszeichen
\newcommand{\gqq}[1]{\glqq #1\grqq}
% Definition von \gq{}: Text in Anführungszeichen
\newcommand{\gq}[1]{\glq #1\grq}
% Spezielle Hervorhebung von Schlüsselwörtern
\newcommand{\textOrdner}[1]{\texttt{#1}}
\newcommand{\textVariable}[1]{\texttt{#1}}
\newcommand{\textKlasse}[1]{\texttt{#1}}
\newcommand{\textFunktion}[1]{\texttt{#1}}
\newcommand{\newparagraph}{\newline \newline}
% Quellenangabe bei Bildern
\newcommand{\customcaption}[2]{\caption[#1]{ #1. #2.}}

%%%%%%%%%%%%%%%%%%%% Zitatbibliothek einbinden %%%%%%%%%%%%%%%%%%%%%
\addbibresource{./literatur/literatur.bib}


%%%%%%%%%%%%%%%%%%%%%%%% PDF-Einstellungen %%%%%%%%%%%%%%%%%%%%%%%%%
\hypersetup{
  bookmarksopen=false,
	bookmarksnumbered=true,
	bookmarksopenlevel=0,
  pdftitle=\vTitel,
  pdfsubject=\vTitel,
  pdfauthor=\vAutor,
  pdfborder={0 0 0},
	pdfstartview=Fit,
  pdfpagelayout=SinglePage
}


%%%%%%%%%%%%%%%%%%%%%%%% Kopf- und Fußzeile %%%%%%%%%%%%%%%%%%%%%%%%
\pagestyle{fancy}
\setlength{\headheight}{15pt}
\fancyhf{}
\fancyhead[R]{\thepage}


%%%%%%%%%%%%%%%%%%%%%%%%%%%%%% Layout %%%%%%%%%%%%%%%%%%%%%%%%%%%%%%
\onehalfspacing
\setlist{noitemsep}

\addto\captionsngerman{
  \renewcommand{\figurename}{Abb.}
  \renewcommand{\tablename}{Tab.}
}
\numberwithin{table}{section}                               % Tabellennummerierung je Sektion zurücksetzen
\numberwithin{figure}{section}                              % Abbildungsnummerierung je Sektion zurücksetzen
\renewcommand{\thetable}{\arabic{section}.\arabic{table}}   % Tabellennummerierung mit Section
\renewcommand{\thefigure}{\arabic{section}.\arabic{figure}} % Abbildungsnummerierung mit Section
\renewcommand{\thefootnote}{\arabic{footnote}}              % Sektionsbezeichnung von Fußnoten entfernen

\renewcommand{\multfootsep}{, }                             % Mehrere Fußnoten durch ", " trennen


%%%%%%%%%%%%%%%%%%%%%%%%%%%%% Dokument %%%%%%%%%%%%%%%%%%%%%%%%%%%%%

\begin{document}


  %%%%%%%%%%%%%%%%%%% Einführung und Verzeichnisse %%%%%%%%%%%%%%%%%%%
  \pagenumbering{Roman}

  \begin{titlepage}
  \begin{minipage}{6in}
    \vspace*{-2cm}
    \centering
    \hspace{-2cm}
	\ifx\vFirmenlogoPfad\empty
	\else
    \raisebox{-0.5\height}{\includegraphics[height=4cm]{\vFirmenlogoPfad}}
  \fi
	\hfill
	\ifx\vDHBWLogoPfad\empty
	\else
   	\raisebox{-0.5\height}{\includegraphics[height=4cm]{\vDHBWLogoPfad}}
	\fi
  \end{minipage}
  \begin{center}
    \vspace*{0.5cm}
    \Huge\textbf{\vTitel}\\
		\ifx\vUntertitel\empty
		\else
			\Large\rm\vUntertitel\\
		\fi
		\vspace*{2cm}
		\Large\textbf{\vArbeitstyp}
		\ifx\vArbeitsbezeichnung\empty
		\else
			\textbf{\vArbeitsbezeichnung}
		\fi
		\\
		\normalsize
		über die \vPhasenbezeichnung\ des \vStudienJahr{n}\ Studienjahrs \\
		\vspace*{1cm}
		an der Fakultät für \vFakultaet\\
		im Studiengang \vStudiengang\\
		\vspace*{0.5cm}
		an der DHBW \vDHBWStandort\\
		\ifx\vDHBWCampus\empty
		\else
		Campus \vDHBWCampus\\
		\fi
		\vspace*{0.5cm}
		von\\
		\ifx\vAutor\empty
		\else
			\vAutor\\
		\fi
		\vspace*{1cm}
		\vAbgabedatum
		\vfill
  \end{center}
  \begin{tabular}{ll}
    Bearbeitungszeitraum:&\vBearbeitungszeitraum\\
    Kurs:&\vKurs\\
	  Dozent der Hochschule:&\vBetreuer\\
  \end{tabular}
\end{titlepage}
\newpage
\setcounter{page}{2}
  % \thispagestyle{empty}
\section*{\Huge{Sperrvermerk}}

\addcontentsline{toc}{section}{Sperrvermerk}
gemäß Ziffer 1.1.13 der Anlage 1 zu §§ 3, 4 und 5  der Studien- und Prüfungsordnung für die Bachelorstudiengänge im Studienbereich Technik der Dualen Hochschule Baden-Würt­tem­berg vom 29.09.2017.\\

\noindent \gqq{Der Inhalt dieser Arbeit darf weder als Ganzes noch in Auszügen Personen außerhalb des Prüfungsprozesses und des Evaluationsverfahrens zugänglich gemacht werden, sofern keine anders lautende Genehmigung vom Dualen Partner vorliegt.}

\vfill
\leavevmode
\newline
\parbox{6cm}{\strut\centering \vBearbeitungsort, \vAbgabedatum\hrule\strut\centering\footnotesize Ort, Datum} 
\hfill
\ifx\vUnterschrift\empty
\parbox{6cm}{\strut\hspace{1pt} \vAbteilung\hrule\strut\centering\footnotesize Abteilung, Unterschrift}
\else
\parbox{6cm}{\strut\hspace{1pt} \vAbteilung, \parbox[b]{3cm}{\vspace{-10cm}\includegraphics[width=3cm]{\vUnterschrift}}\hrule\strut\centering\footnotesize Abteilung, Unterschrift}
\fi
\vspace{1cm}

\newpage
  \thispagestyle{empty}
\section*{\Huge{Gender Erklärung}}

\addcontentsline{toc}{section}{Gendererklärung}
Aus Gründen der besseren Lesbarkeit wird in dieser Bachelorarbeit auf die gleichzeitige Verwendung der Sprachformen männlich,
weiblich und divers (m/w/d) verzichtet. Sämtliche Formulierungen gelten gleichermaßen für alle Geschlechter.
\newpage
  \thispagestyle{empty}
\section*{\Huge{Selbstständigkeitserklärung}}

\addcontentsline{toc}{section}{Selbstständigkeitserklärung}
gemäß Ziffer 1.1.13 der Anlage 1 zu §§ 3, 4 und 5  der Studien- und Prüfungsordnung für die Bachelorstudiengänge im Studienbereich Technik der Dualen Hochschule Baden-Würt­tem­berg vom 29.09.2017.

\noindent Wir versichern hiermit, dass wir unsere Bachelorarbeit (bzw. Projektarbeit oder Studienarbeit bzw. Hausarbeit) mit dem Thema: 
\begin{center}
	\Large\textbf{\vTitel}
\end{center}
selbstständig verfasst und keine anderen als die angegebenen Quellen und Hilfsmittel benutzt haben. Wir versichern zudem, dass die eingereichte elektronische Fassung mit der gedruckten Fassung übereinstimmt.

\vfill
\leavevmode
\newline
\parbox{7cm}{\strut\centering \vBearbeitungsort, \vAbgabedatum\hrule\strut\centering\footnotesize Ort, Datum} 
\hfill
\parbox{7cm}{\strut\hspace{1pt} \hrule\strut\centering\footnotesize \vJB}
\newline
\vspace{1cm}
\newline
\parbox{7cm}{\strut\centering \vBearbeitungsort, \vAbgabedatum\hrule\strut\centering\footnotesize Ort, Datum} 
\hfill
\parbox{7cm}{\strut\hspace{1pt} \hrule\strut\centering\footnotesize \vLB}
\newline
\vspace{1cm}
\newline
\parbox{7cm}{\strut\centering \vBearbeitungsort, \vAbgabedatum\hrule\strut\centering\footnotesize Ort, Datum} 
\hfill
\parbox{7cm}{\strut\hspace{1pt} \hrule\strut\centering\footnotesize \vPP}
\newline
\vspace{1cm}
\newline
\parbox{7cm}{\strut\centering \vBearbeitungsort, \vAbgabedatum\hrule\strut\centering\footnotesize Ort, Datum} 
\hfill
\parbox{7cm}{\strut\hspace{1pt} \hrule\strut\centering\footnotesize \vHS}
\newpage
  %\phantomsection
\newenvironment{keywords}{
	\begin{flushleft}
	\small	
	\textbf{
		\iflanguage{ngerman}{Schlüsselwörter}{\iflanguage{english}{Keywords}{}}
	}
}{\end{flushleft}}

% Deutsche Zusammenfassung
\begin{abstract}
	
\end{abstract}

% Schlüsselwörter Deutsch
\begin{keywords}
	
\end{keywords}


\selectlanguage{english}
% Englisches Abstract
\begin{abstract}

\end{abstract}

% Schlüsselwörter Englisch
\begin{keywords}

\end{keywords}


\selectlanguage{ngerman}
\newpage
  \pdfbookmark[1]{\contentsname}{toc}
\tableofcontents
\newpage
  \section*{Abkürzungsverzeichnis}
\addcontentsline{toc}{section}{Abkürzungsverzeichnis}
\begin{acronym}
  \acro{DHBW}[DHBW]{Duale Hochschule Ba\-den-\-Würt\-tem\-berg}
  \acroplural{DHBW}[DHBW]{Dualen Hochschule Ba\-den-\-Würt\-tem\-berg}
  \acro{ISO}[ISO]{International Organization for Standardization}
  \acro{RE}[RE]{Requirements Engineering}
  \acro{FMEA}[FMEA]{Failure Mode and Effects Analysis}
  \acro{KVP}[KVP]{Kontinuierlicher Verbesserungs Prozess}
  \acro{CMMI}[CMMI]{Capability Maturity Model Integration}
\end{acronym}
\newpage
  \listoffigures
\newpage
  \listoftables
\newpage
  \lstlistoflistings
\addcontentsline{toc}{section}{Listings}
\newpage
  % \section*{Vorwort}
\addcontentsline{toc}{section}{Vorwort}
\newpage


  %%%%%%%%%%%%%%%%%%%%%%%%%%%%% Kapitel %%%%%%%%%%%%%%%%%%%%%%%%%%%%%%
  \pagestyle{fancy}
  \fancyhead[L]{\nouppercase{\rightmark}}    % Abschnittsname im Header
  \pagenumbering{arabic}

  %%%%%%%%%%%%%%%%%%%%%%%%%%%%%%%%%%%%%%%%%%%%%%%%%%%%%%%%%%%%%%%%%%%%
  %%%%                   EIGENE KAPITEL EINFÜGEN                  %%%%
  %%%%%%%%%%%%%%%%%%%%%%%%%%%%%%%%%%%%%%%%%%%%%%%%%%%%%%%%%%%%%%%%%%%%
  \section{Einleitung}
Die Hausarbeit wird im Rahmen der Vorlesung Softwarequalität erstellt.
Dabei werden zunächst Grundwissen und Begriffe definiert, anschließend werden die erlernten Fähigkeiten auf eine konkretes Produkt angewandt.
Hierzu wird das Softwareprodukt \gqq{DEV-CHAT} aus der Vorlesung Software Engineering 1 verwendet.
\newparagraph
%Beschreibung devchat
DEV-CHAT ist ein Chatportal mit mehreren Chatrooms, denen man über einen Schlüssel beitreten kann.
Die Oberfläche ist schlicht im Informatik/Konsolen Stil gehalten und alle Operationen werden über eine Kommandozeile ausgeführt.
Über die Kommandozeile lassen sich Befehle ausführen, um zum Beispiel Direktnachrichten zu senden oder Umfragen zu starten.
\newparagraph
%Aufgabenstellung
Zu Begin der Hausarbeit werden, wie bereits angekündigt, die Begriffe Softwarequalität und Qualität definiert und gegeneinander abgegrenzt, anschließend wird das Kano-Modell erläutert und begründet wieso Requirements Engineering essenziell für eine hohe Softwarequalität ist.
Die Folgeschritte werden auf den DEV-CHAT angewandt, diese gliedern sich in drei Punkte.
Zunächst werden die Anforderungen und Nutzerwartungen definiert, in einem weiteren Schritt werden Fehler, Kosten und Qualitätsmaßnahmen behandelt.
Abschließend wird das Thema Umsetzung und Tests erörtert.
  \section{Grundwissen und Begriffe}
In diesem Kapitel werden die Grundlegenden Begriffe und Konzepte vorgestellt und erläutert.
Dazu wird zunächst der generische Begriff der Qualität definiert und anschließend gegenüber der Software-Qualität abgegrenzt.
Weiterhin wird das Kano-Modell vorgestellt und dessen Bedeutung in Zusammenhang mit dem Requirements Engineering in Bezug auf die Software-Qualität erläutert.
\subsection{Qualität}
Der Begriff Qualität leitet sich von dem lateinischen Begriff \gqq{qualitas} ab, welcher mit den Begriffen \gqq{Beschaffenheit} oder \gqq{Eigenschaft} übersetzt werden kann \autocite[vgl.][]{noauthor_was_nodate}.
Die \ac{ISO} beschreibt den Begriff Qualität als \gqq{Grad, in dem ein Satz inhärenter Merkmale eines Objekts Anforderungen erfüllt} \autocite[S. 17]{iso_iso_2015}.
Die Merkmale, sowie Anforderungen des Objekts werden dabei nicht weiter konkretisiert.
Es handelt sich also um Aspekte, welche nicht universell definiert werden können.
Qualität ist somit ein sehr abstrakter Begriff, welcher je nach Kontext unterschiedlich interpretiert werden kann.

Um eine Aussage über die Qualität eines Objekts tätigen und mit anderen Objekten vergleichen zu können, muss Qualität messbar gemacht werden.
Dafür müssen zunächst die wesentlichen gemeinsamen Merkmale definiert werden.
In einem weiteren Schritt muss dann eine Art und Weise gefunden werden, wie diese spezifischen Merkmale hinsichtlich ihres Umsetzungsgrades bewertet werden können.
Die Bewertung kann dabei sowohl subjektiv als auch objektiv erfolgen \autocite[vgl.][S. 53]{shewhart_economic_1931}.
Das Merkmal Benutzerfreundlichkeit kann beispielsweise durch die Auswertung verschiedener subjektiver Eindrücke von Test-Benutzer bestimmt werden.
Gleichzeitig kann aber auch eine objektive Bewertung vorgenommen werden, indem beispielsweise Faktoren wie die Anzahl an Funktionen zusammen mit der Komplexität der einzelnen Funktionen des Objekts gemessen werden.

\subsection{Software-Qualität}
Der Begriff Software-Qualität stellt eine Spezifizierung des Begriffs Qualität dar, indem der Begriff Qualität in den Kontext Software gestellt wird.
Durch diese Spezifizierung konkretisiert sich auch die Definition des Begriffs (Software-)Qualität, welcher als "Grad, in dem ein Softwareprodukt festgelegte und implizierte Anforderungen erfüllt, wenn es unter bestimmten Bedingungen verwendet wird" \autocite[S. 17]{iso_iso_2011}, definiert werden kann.
Die wesentlichen Merkmale für die Bewertung der Qualität von Software unterteilen sich dabei in zwei Gruppen -- die externe und die interne Qualitätssicht.

Die externe Qualitätssicht bezieht sich auf Software-Merkmale, welche für die anwendende Person die Qualität der Software beeinflusst.
Dazu zählen die vier Merkmale Funktionalität, Laufzeit, Zuverlässigkeit und Benutzbarkeit.

Auf der anderen Seite steht die interne Qualitätssicht, welche sich auf Software-Merkmale, welche für die entwickelnde Person relevant sind, bezieht.
Dazu zählen die Merkmale Wartbarkeit, Transparenz, Übertragbarkeit sowie Testbarkeit \autocite[vgl.][S. 6-10]{hoffmann_software-qualitat_2013}.

\subsection{Kano-Modell}
Damit ein entwickeltes Produkt einen möglichst hohen Gewinn generieren kann, muss das Produkt möglichst oft abgesetzt werden können.
Dies kann unter anderem durch eine Steigerung der Kundenzufriedenheit erreicht werden.
Um diese systematisch zu erreichen, entwickelte der japanische Professor Noriaki Kano in den 70er Jahren das Kano-Modell \autocite[vgl.][S. 27]{sauerwein_kano-modell_2000}.
Kano definiert dafür fünf Kategorien, in welche sich die Merkmale eines Produktes einteilen lassen: 
Basismerkmale (Must-be quality elements), Leistungsmerkmale (One-dimensional quality elements), Begeisterungsmerkmale (Attractive quality elements), Unerhebliche-Merkmale (Indifferent quality elements) und Rückweisungsmerkmale (Reverse quality elements) \autocite[vgl.][S. 82-83]{holzing_kano-theorie_2008}.

\begin{figure}
  \centering
  \includegraphics[width=0.8\textwidth, keepaspectratio]{images/kano-modell.png}
  \caption{Kano-Modell \autocite{diehl_kano_2019}}
  \label{fig:kano-modell}
\end{figure}
Die einzelnen Kategorien unterscheiden sich hinsichtlich ihrer Wirkung auf die Kundenzufriedenheit.
Abbildung~\ref{fig:kano-modell} zeigt drei der fünf Kategorien des Kano-Modells.
Die x-Achse definiert dabei den Grad der Erfüllung der Merkmale, wobei die y-Achse den Grad der Kundenzufriedenheit darstellt.

Die rote Kurve in Abbildung~\ref{fig:kano-modell} zeigt die Charakteristik der Basismerkmale.
Während eine unzureichende Erfüllung des Merkmals zu einer negativen Kundenzufriedenheit führt, führt eine übermäßige Erfüllung des Merkmals nicht zu einer positiven Kundenzufriedenheit, da diese Merkmale als absolutes Minimum (Must-be) angesehen werden.

Im Gegensatz dazu stehen die Leistungsmerkmale, welche durch die blaue Kurve charakterisiert werden.
Hierbei handelt es sich um einen linearen Zusammenhang zwischen dem Grad der Erfüllung eines Merkmals und der Kundenzufriedenheit.
Durch Erfüllung dieser Merkmale kann somit im Gegensatz zu den Basismerkmalen die Kundenzufriedenheit gesteigert werden.

Begeisterungsmerkmale stellen das Komplement zu den Basismerkmalen dar und sind in Abbildung~\ref{fig:kano-modell} durch die rosa Kurve dargestellt.
Dabei handelt es sich um optionale Merkmale, welche die Kundenzufriedenheit ausschließlich steigern können.
Werden die Merkmale nicht erfüllt beeinflusst dies die Kundenzufriedenheit somit weder positiv noch negativ.

Neben den drei eingezeichneten Kurven in Abbildung~\ref{fig:kano-modell} wird die Klasse der Unerheblichen-Merkmale durch die x-Achse charakterisiert.
Diese Merkmale sind für die Kundenzufriedenheit nicht relevant, da sie diese weder positiv noch negativ beeinflussen.

Die letzte Klasse, die Rückweisungsmerkmale, stellen das Komplement zu den Leistungsmerkmalen dar.
Es besteht somit ebenfalls ein linearer Zusammenhang zwischen dem Grad der Erfüllung eines Merkmals und der Kundenzufriedenheit, jedoch in umgekehrter Richtung.
In Abbildung~\ref{fig:kano-modell} würde diese Klasse der blauen Geraden entsprechen, welche um 90 Grad im Ursprung gedreht wurde.
Bei nicht Erfüllung des Merkmals wird somit eine hohe Kundenzufriedenheit hervorgerufen, wohingegen eine Erfüllung des Merkmals die Kundenzufriedenheit negativ beeinflusst.
\newline

% TODO: @Henry Zusammenhang Qualität und Kano-Modell ?

\subsection{Requirements Engineering}
\ac{RE} ist ein Teilgebiet der Software-Entwicklung, welches sich mit der Erhebung, Analyse und Spezifikation von Anforderungen an ein Software-Produkt befasst.
Diese Anforderungen werden dabei durch die Stakeholder des Software-Produkts definiert.
Insgesamt lassen sich die Anforderungen in drei Kategorien unterteilen: funktionale Anforderungen, Qualitätsanforderungen und Randbedingungen \autocite[vgl.][S. 3]{pohl_basiswissen_2021}.
\newparagraph
Bereits aus den Kategorien der Anforderungen lässt sich der Zusammenhang mit der Softwarequalität ableiten.
Qualitätsanforderungen sind demnach fester Bestandteil des \ac{RE}, da diese die Systemarchitektur maßgeblich beeinflussen können \autocite[vgl.][S. 3-4]{pohl_basiswissen_2021}.
\citeauthor{ebert_systematisches_2019} nennt als Folge eines unzureichenden \ac{RE} eine hohe Zufälligkeit der resultierenden Softwarequalität, da die zu erreichenden Ziele nicht klar definiert sind \autocite[vgl.][S. 51]{ebert_systematisches_2019}. 

Das frühe Einbringen der Qualitätsanforderungen bereits im Prozess des \ac{RE} ist dabei essenziell für die resultierenden Qualitätsmerkmale.
\citeauthor{ebert_systematisches_2019} kommt zu dem Schluss, dass Unternehmen erkannt haben, dass \gqq{Qualität nur dann effektiv und kostengünstig erreichbar ist, wenn sie bereits zu Beginn des Projekts in dessen Zentrum steht - und nicht am Ende \gq{hineingetestet} werden soll} \autocite[S. 66]{ebert_systematisches_2019}.
Durch dieses Vorgehen wird die Anzahl an Änderungen im späteren Projektverlauf reduziert, da die Anforderungen bereits im \ac{RE} definiert wurden und somit nicht mehr verändert werden müssen.

% TODO: @Henry Eventuell RE noch genauer beschreiben? -> Optional
  \include{chapter/FehlerKostenQualitaetsmaßnahmen}
  \section{Umsetzung und Tests}
In diesem Kapitel wird auf die Umsetzung und Teststrategie eines Software Produkts eingegangen.
Hierzu wird zunächst eine Aufwandsschätzung für die Umsetzung einer Neuentwicklung des \gqq{DEV-CHAT} durchgeführt.
In einem weiteren Schritt wird ein risikobasierter Testplan erstellt und eine Teststrategie entwickelt.
Abschließend werden exemplarisch zwei Testfälle ausformuliert und durchgeführt.

\subsection{Aufwandsschätzung einer Neuentwicklung}
Da es sich bei einem Software Produkt nicht um ein herkömmliches materielles Produkt handelt, gestaltet sich die Aufwandsschätzung für die Umsetzung eines Software Produkts schwierig.
Besonders der Entwicklungsfortschritt und der Grad der Fertigstellung lassen sich nur schwer messen.
Aus diesem Grund wurden verschiedene Modelle entwickelt, die eine Aufwandsschätzung ermöglichen.
Im Rahmen dieser Arbeit werden das \ac{COCOMO}, der Netzplan und die \ac{FPA} genauer betrachtet.

\subsubsection{Constructive Cost Model}
% COCOMO
% 8 KDSI -> PM = 2,4 * KDSI^1,05 -> 2,4 * 8^1,05 = 21,3
% TDEV = 2,5 * PM^0,38 -> 2,5 * 16,8^0,38 = 8
% Personalbedarf: 21,3 / 8 = 2,7
Das \ac{COCOMO} bewertet den Aufwand einer Software anhand der Anzahl der benötigten Programmzeilen.
Im Gegensatz zu dem Entwicklungsaufwand selbst, lassen sich die benötigten Programmzeilen leicht abschätzen.
Trotzdem birgt das Verfahren einige Schwächen.
So ist die Abschätzung der benötigten Programmzeilen mit einem hohen Aufwand verbunden und erfordert bereits zu Beginn des Projekts eine technologiespezifische Expertise.
Außerdem werden agile Entwicklungsprozesse nicht berücksichtigt, weshalb sich \ac{COCOMO} besonders in Kombination mit dem Wasserfallmodell anbietet.

\noindent{}Um das \ac{COCOMO} durchführen zu können, muss zunächst die Komplexität des Produkts bestimmt werden.
Diese ergibt sich, wie bereits erwähnt, durch die Anzahl der benötigten Programmzeilen, welche in der Einheit \ac{KDSI} angegeben werden.
Da es sich hier um eine Neuentwicklung handelt, wird die Komplexität der Neuentwicklung in etwa der Komplexität des \gqq{DEV-CHAT} (8~\ac{KDSI}) entsprechen.
Folglich ist das Produkt in die kleinste Komplexitätsstufe, den sogenannten \gqq{Organic Mode}, einzuordnen.

\begin{equation}
  \begin{split}
    \text{\acs{PM}} &= 2,4 \cdot \text{\acs{KDSI}}^{1,05} \\
    \text{\acs{TDEV}} &= 2,5 \cdot \text{\acs{PM}}^{0,38}
  \end{split}
  \label{eq:cocomo}
\end{equation}
\noindent{}Aus diesen Informationen lässt sich nun der Aufwand, gemessen in \ac{PM} ($1~\ac{PM} \equiv$ 19 Arbeitstage zu je 8 Stunden), sowie die \ac{TDEV} berechnen.
Dazu werden die Formeln aus \eqref{eq:cocomo} verwendet.
Es ergibt sich ein Aufwand von ca. 21,3~\ac{PM} ($\approx 3200$ Stunden).
Daraus resultiert eine \ac{TDEV} von ca. 8 Monaten mit einem Personalbedarf von $\frac{21,3}{8} \approx 3$ Personen.

\subsubsection{Netzplan}
% Netzplan

\subsubsection{Function Point Analyse}
% FPA
Bei der \ac{FPA} werden die Funktionalen Anforderungen eines Produkts genauer betrachtet und in kleinstmögliche, sinnvolle Aktivitäten zerlegt.
Jedem Elementarprozess wird anschließend ein definierter Punktewert, gemessen in \ac{fp}, zugewiesen.
Die Summe aller Einzelwerte ergibt die Functional-Size des Produkts.

\begin{table}
  \centering
  \begin{tabular}{|l|c|}
    \hline
    \textbf{Elementarprozessart} & \textbf{Punktewert} in \acs{fp} \\
    \hline
    \ac{EI} & \EI \\
    \hline
    \ac{EO} & \EO \\
    \hline
    \ac{EQ} & \EQ \\
    \hline
    \ac{ILF} & \ILF \\
    \hline
    \ac{EIF} & \EIF \\
    \hline
  \end{tabular}
  \caption{Punktewerte nach der Rapid-Näherung}
  \label{tab:rapidNaeherung}
\end{table}
\noindent{}Bei den Elementarprozessen wird zwischen fünf verschiedenen Arten unterschieden.
Tabelle \ref{tab:rapidNaeherung} zeigt die verschiedenen Arten und ihre jeweiligen Punktewerte nach der Rapid-Näherung.

\begin{table}
  \centering
  \begin{tabular}{|l|c|c|}
    \hline
    \textbf{Elementarprozess} & \textbf{Typ} & \textbf{Punktewert} in \acs{fp} \\
    \hline
    Benutzer Datenbank & \acs{ILF} & \ILF \\
    \hline
    Login & \acs{EI} & \EI \\
    \hline
    Registrieren & \acs{EI} & \EI \\
    \hline
    Passwort ändern & \acs{EI} & \EI \\
    \hline
    Chat Datenbank & \acs{ILF} & \ILF \\
    \hline
    Raum erstellen & \acs{EI} & \EI \\
    \hline
    Nachrichten erstellen & \acs{EI} & \EI \\
    \hline
    Nachrichten lesen/synchronisieren & \acs{EQ} & \EQ \\
    \hline
    Umfrage Datenbank & \acs{ILF} & \ILF \\
    \hline
    Kurzbefehle & \acs{EI} & \EI \\
    \hline
    Umfrage erstellen & \acs{EI} & \EI \\
    \hline
    Umfrage ausgeben & \acs{EQ} & \EQ \\
    \hline
    Umfrage abstimmen & \acs{EI} & \EI \\
    \hline
    Rechnung ausführen & \acs{EI} & \EI \\
    \hline
    Ticket Datenbank & \acs{ILF} & \ILF \\
    \hline
    Ticket erstellen & \acs{EI} & \EI \\
    \hline
    Benutzer anzeigen & \acs{EQ} & \EQ \\
    \hline
    Benutzer löschen & \acs{EI} & \EI \\
    \hline
    Benutzer Passwort zurücksetzen & \acs{EI} & \EI \\
    \hline
    Benutzer Rolle vergeben & \acs{EI} & \EI \\
    \hline
    Custom-Chat-Raum erstellen & \acs{EI} & \EI \\
    \hline
    Chat-Raum/Umfrage/Tickets löschen & \acs{EI} & \EI \\
    \hline
    Chat-Raum/Umfrage Ablaufdatum setzen & \acs{EI} & \EI \\
    \hline
    \hline
    \multicolumn{2}{|l|}{\textbf{Summe der Elementarprozessbewertungen}} & \textbf{104} \\ % 16*EI + 3*EQ + 4*ILF
    \hline
  \end{tabular}
  \caption{Funktion Point Analyse der Neuentwicklung}
  \label{tab:FPA}
\end{table}
\noindent{}Basierend auf dieser Bewertungsskala wird die \ac{FPA} für die Neuentwicklung des \gqq{DEV-CHAT} in Tabelle~\ref{tab:FPA} durchgeführt.
Insgesamt ergibt sich eine Punktzahl von 104~\ac{fp}, was einem Zeitaufwand von ca. 1000 Stunden für die Neuentwicklung des \gqq{DEV-CHAT} entspricht.

\subsubsection{Fazit}
% Fazit
% COCOMO: 21,3 PM (3200 Stunden)
% Netzplan: 
% FPA: 1000 Stunden

\subsection{Testplan und Teststrategie}


\subsection{Exemplarische Durchführung}

  %%%%%%%%%%%%%%%%%%%%%%% Literaturverzeichnis %%%%%%%%%%%%%%%%%%%%%%%
  \phantomsection
\addcontentsline{toc}{section}{Literatur}
\printbibliography
\newpage


  %%%%%%%%%%%%%%%%%%%%%%%%%%%%%% Anhang %%%%%%%%%%%%%%%%%%%%%%%%%%%%%%
  \renewcommand{\thetable}{\Alph{section}.\arabic{table}}
  \renewcommand{\thefigure}{\Alph{section}.\arabic{figure}}
  \renewcommand{\thelstlisting}{\Alph{section}.\arabic{lstlisting}}
  \pagenumbering{Alph}

  \begin{appendix}
  \section{Anhang}
\end{appendix}
\end{document}