\section{Fehler, Kosten und Qualitätsmaßnahmen}

% kurze Einleitung zum Thema
In diesem Kapitel wird auf die Fehler, Kosten und Qualitätsmaßnahmen eingegangen.
Hierzu werden typische Faults und Defects des Produkts beschrieben, diese werden durch Fehlerschwere, Priorität und potentielle Fehlerkosten klassifiziert. 
Anschließend werden mehrere Methoden dargestellt, wie die Qualitätskosten in der Entwicklung gesenkt werden können.
Außerdem wird erörtert warum es nicht sinnvoll ist eine vollständige Fehlersicherheit anzustreben.
Abschließend wird für das Produkt eine \ac{FMEA} durchgeführt.

\subsection{Faults und Defects}
% Beschreiben Sie typische Faults and Defects Ihres Software-Produkts. Stellen Sie Fehlerschwere, Priorität und potentielle Fehlerkosten dar.
% Notizen: 
    % Error: Fehlhandlung (Menschliche Handlung die zu einem nicht korrektem Resultat führt Beispiel: Programmierer – Fehler in Programmcode.)
    % Fault: Fehlerzustand (Auftreten eines Fehlers in der Software.)
    % Failure (/Defect): Fehlerwirkung (Bei der Programmausführung kommt es zu einem Abbruch  oder bestimmte Funktionen können nicht wie spezifiziert ausgeführt werden.)

% Aufgaben Aufbau:
    % Faults und Defects [] (-> Tabelle?)
        % Fehlerschwere (1-5) (Prio auch noch mit rein?)
        % Priorität
        % potentielle Fehlerkosten




\subsection{Qualitätskostensenkung und Fehlertoleranz}
%Stellen Sie dar, wie sie die Q-Kosten in der Entwicklung des Produktes senken können? Warum ist es nicht sinnvoll komplette Fehlerfreiheit anzustreben und nennen Sie Methoden und Modelle aus der Vorlesung, die zum Einsatz kommen könnten?
Grundsätzlich wird Software in den klassischen Vorgehensmodellen innerhalb eines Projekts in den sechs Phasen Analyse, Entwurf, Implementierung, Test, Inbetriebnahme und Wartung entwickelt.
Bei agilen Prozessmodellen werden diese Phasen wiederholt durchlaufen, während im Wasserfallmodel oder V-Modell diese Phasen im Idealfall nur einmal durchlaufen werden.
Das Ziel ist, dass bereits in der Analyse alle Anforderungen an die zu entwickelde Software identifizert werden.
Änderungen die nach dem Entwurf der Software entstehen können meist nur mit hohem Kostenaufwand berücksichtigt werden.
Daraus ergibt sich die auf Barry Boehm zurückgehende Kostenkurve die in Abb. \ref{fig:boehm} dargestellt wird.
Diese stellt dar, dass später auftretende Änderungen exponentiell höhere Kosten verursachen.
\autocite[vgl.][S. 95]{witte_testmanagement_2019}
\begin{figure}[H]
    \centering
    \includegraphics[width=1\textwidth]{images/boehm.png}
    \customcaption{Kostenkurve Barry Boehm}{\autocite[][S. 97]{witte_testmanagement_2019}}
    \label{fig:boehm}
\end{figure}\noindent
% evtl 10er regel / defektvererbung
Um Qualitätsbezogene Kosten zu Reduzieren muss zunächst die Zusammensetzung dieser dargestellt werden.
Qualitätskosten setzen sich aus den den drei Bereichen Fehlerkosten, Fehlerverhütungskosten und Prüfkosten zusammen.
Fehlerkosten sind Kosten, die nach der Auslieferung des Produkts entstehen, wie z.B. Kosten für Reklamationen oder Vertragsstrafen.
Fehlerverhütungskosten sind Kosten die primär im Bereich des Qualitätsmanagement anfallen.
Prüfkosten entstehen in der Entwicklung bzw. Produktion, um das Produkt vor der Auslieferung zu Testen.
Die Fehlerverhütungskosten und Prüfkosten können zusammengefasst werden, das diese während der Entwicklung entstehen, die Fehlerkosten hingegen entstehen nach der Auslieferung des Produkt.
\newparagraph
Betrachtet man nun die Fehlerkosten in Relation zu den Fehlerverhütungs- und Prüfkosten lässt sich ein Zusammenhang erkennen.
Wird während der Entwicklung Wert auf eine hohe Qualität gelegt, also höhere Qualitätskosten im Bereich Fehlerverhütungs- und Prüfkosten, um so weniger Fehlerkosten entstehen, da durch die hohe Qualität mit weniger Beschwerden am ausgelieferten Produkt gerechnet werden kann.
Dieser Zusammenhang wird in der Abbildung \ref{fig:QKostenVerhaeltnis} dargestellt.
\begin{figure}[H]
    \centering
    \includegraphics[width=1\textwidth]{images/qkostenverhaeltnis.png}
    \caption{Qualitätskosten Verhältnis}
    \label{fig:QKostenVerhaeltnis}
\end{figure}\noindent
% Wie Qualitätskosten in der Entwicklung senken?
% - S.28
% Warum nicht komplette Fehlerfreiheit?
Um die Qualitätskosten in Summe zu Reduzieren muss das perfekte Mittel zwischen den Fehlerverhütungs- und Prüfkosten und den Fehlerkosten gefunden werden.
Unter Berücksichtigung dieses Zusammenhangs wird ebenfalls klar, dass eine vollständige Fehlerfreiheit aus Kostensicht nicht sinnvoll ist.
Der Aufwand im Bereich der Fehlerverhütungskosten und Prüfkosten wächst exponentiell im Verhältnis zur Qualität während sich die Fehlerkosten nur langsam der Nulllinie annähern. 
% Die Kostenoptimierung besteht aus 3 Teilen.
%   Erfassung q-Kosten
%   Prozessverbesserung --> Verbesserung d. Entwicklungsprozesses
%   Produktverbesserung --> Verbesserung d. tatsächlichen Produkts
% Reduzierung durch PDCA / Six Sigma / CMMI Dev

% PDCA / CMMI-DEV --> Prozessoptimierung
% 



\subsection{\acl{FMEA}}

% Failure Mode and Effects Analysis (FMEA) (realistische Annahme für Risikoprioritätszahl RPZ




