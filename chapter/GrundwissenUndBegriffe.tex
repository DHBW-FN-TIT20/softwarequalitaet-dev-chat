\section{Grundwissen und Begriffe}
In diesem Kapitel werden die Grundlegenden Begriffe und Konzepte vorgestellt und erläutert.
Dazu wird zunächst der generische Begriff der Qualität definiert und anschließend gegenüber der Software-Qualität abgegrenzt.
Weiterhin wird das Kano-Modell vorgestellt und dessen Bedeutung in Zusammenhang mit dem Requirements Engineering in Bezug auf die Software-Qualität erläutert.
\subsection{Qualität}
Der Begriff Qualität leitet sich von dem lateinischen Begriff \gqq{qualitas} ab, welcher mit den Begriffen \gqq{Beschaffenheit} oder \gqq{Eigenschaft} übersetzt werden kann \autocite[vgl.][]{noauthor_was_nodate}.
Die \ac{ISO} beschreibt den Begriff Qualität als \gqq{Grad, in dem ein Satz inhärenter Merkmale eines Objekts Anforderungen erfüllt} \autocite[S. 17]{iso_iso_2015}.
Die Merkmale, sowie Anforderungen des Objekts werden dabei nicht weiter konkretisiert.
Es handelt sich also um Aspekte, welche nicht universell definiert werden können.
Qualität ist somit ein sehr abstrakter Begriff, welcher je nach Kontext unterschiedlich interpretiert werden kann.

Um eine Aussage über die Qualität eines Objekts tätigen und mit anderen Objekten vergleichen zu können, muss Qualität messbar gemacht werden.
Dafür müssen zunächst die wesentlichen gemeinsamen Merkmale definiert werden.
In einem weiteren Schritt muss dann eine Art und Weise gefunden werden, wie diese spezifischen Merkmale hinsichtlich ihres Umsetzungsgrades bewertet werden können.
Die Bewertung kann dabei sowohl subjektiv als auch objektiv erfolgen \autocite[vgl.][S. 53]{shewhart_economic_1931}.
Das Merkmal Benutzer:innenfreundlichkeit kann beispielsweise durch die Auswertung verschiedener subjektiver Eindrücke von Test-Benutzer:innen bestimmt werden.
Gleichzeitig kann aber auch eine objektive Bewertung vorgenommen werden, indem beispielsweise Faktoren wie die Anzahl an Funktionen zusammen mit der Komplexität der einzelnen Funktionen des Objekts gemessen werden.

\subsection{Software-Qualität}
Der Begriff Software-Qualität stellt eine Spezifizierung des Begriffs Qualität dar, indem der Begriff Qualität in den Kontext Software gestellt wird.
Durch diese Spezifizierung konkretisiert sich auch die Definition des Begriffs (Software-)Qualität, welcher als "Grad, in dem ein Softwareprodukt festgelegte und implizierte Anforderungen erfüllt, wenn es unter bestimmten Bedingungen verwendet wird" \autocite[S. 17]{iso_iso_2011}, definiert werden kann.
Die wesentlichen Merkmale für die Bewertung der Qualität von Software unterteilen sich dabei in zwei Gruppen -- die externe und die interne Qualitätssicht.

Die externe Qualitätssicht bezieht sich auf Software-Merkmale, welche für die anwendende Person die Qualität der Software beeinflusst.
Dazu zählen die vier Merkmale Funktionalität, Laufzeit, Zuverlässigkeit und Benutzbarkeit.

Auf der anderen Seite steht die interne Qualitätssicht, welche sich auf Software-Merkmale, welche für die entwickelnde Person relevant sind, bezieht.
Dazu zählen die Merkmale Wartbarkeit, Transparenz, Übertragbarkeit sowie Testbarkeit \autocite[vgl.][S. 6-10]{hoffmann_software-qualitat_2013}.

\subsection{Kano-Modell}
Damit ein entwickeltes Produkt einen möglichst hohen Gewinn generieren kann, muss das Produkt möglichst oft abgesetzt werden können.
Dies kann unter anderem durch eine Steigerung der Kundenzufriedenheit erreicht werden.
Um diese systematisch zu erreichen, entwickelte der japanische Professor Noriaki Kano in den 70er Jahren das Kano-Modell \autocite[vgl.][S. 27]{sauerwein_kano-modell_2000}.
Kano definiert dafür fünf Kategorien, in welche sich die Merkmale eines Produktes einteilen lassen: 
Basismerkmale (Must-be quality elements), Leistungsmerkmale (One-dimensional quality elements), Begeisterungsmerkmale (Attractive quality elements), Unerhebliche-Merkmale (Indifferent quality elements) und Rückweisungsmerkmale (Reverse quality elements) \autocite[vgl.][S. 82-83]{holzing_kano-theorie_2008}.

\begin{figure}
  \centering
  \includegraphics[width=0.8\textwidth, keepaspectratio]{images/kano-modell.png}
  \caption{Kano-Modell \autocite{diehl_kano_2019}}
  \label{fig:kano-modell}
\end{figure}
Die einzelnen Kategorien unterscheiden sich hinsichtlich ihrer Wirkung auf die Kundenzufriedenheit.
Abbildung~\ref{fig:kano-modell} zeigt drei der fünf Kategorien des Kano-Modells.
Die x-Achse definiert dabei den Grad der Erfüllung der Merkmale, wobei die y-Achse den Grad der Kundenzufriedenheit darstellt.

Die rote Kurve in Abbildung~\ref{fig:kano-modell} zeigt die Charakteristik der Basismerkmale.
Während eine unzureichende Erfüllung des Merkmals zu einer negativen Kundenzufriedenheit führt, führt eine übermäßige Erfüllung des Merkmals nicht zu einer positiven Kundenzufriedenheit, da diese Merkmale als absolutes Minimum (Must-be) angesehen werden.

Im Gegensatz dazu stehen die Leistungsmerkmale, welche durch die blaue Kurve charakterisiert werden.
Hierbei handelt es sich um einen linearen Zusammenhang zwischen dem Grad der Erfüllung eines Merkmals und der Kundenzufriedenheit.
Durch Erfüllung dieser Merkmale kann somit im Gegensatz zu den Basismerkmalen die Kundenzufriedenheit gesteigert werden.

Begeisterungsmerkmale stellen das Komplement zu den Basismerkmalen dar und sind in Abbildung~\ref{fig:kano-modell} durch die rosa Kurve dargestellt.
Dabei handelt es sich um optionale Merkmale, welche die Kundenzufriedenheit ausschließlich steigern können.
Werden die Merkmale nicht erfüllt beeinflusst dies die Kundenzufriedenheit somit weder positiv noch negativ.

Neben den drei eingezeichneten Kurven in Abbildung~\ref{fig:kano-modell} wird die Klasse der Unerheblichen-Merkmale durch die x-Achse charakterisiert.
Diese Merkmale sind für die Kundenzufriedenheit nicht relevant, da sie diese weder positiv noch negativ beeinflussen.

Die letzte Klasse, die Rückweisungsmerkmale, stellen das Komplement zu den Leistungsmerkmalen dar.
Es besteht somit ebenfalls ein linearer Zusammenhang zwischen dem Grad der Erfüllung eines Merkmals und der Kundenzufriedenheit, jedoch in umgekehrter Richtung.
In Abbildung~\ref{fig:kano-modell} würde diese Klasse der blauen Geraden entsprechen, welche um 90 Grad im Ursprung gedreht wurde.
Bei nicht Erfüllung des Merkmals wird somit eine hohe Kundenzufriedenheit hervorgerufen, wohingegen eine Erfüllung des Merkmals die Kundenzufriedenheit negativ beeinflusst.
\newline

% TODO: @Henry Zusammenhang Qualität und Kano-Modell ?

\subsection{Requirements Engineering}
\ac{RE} ist ein Teilgebiet der Software-Entwicklung, welches sich mit der Erhebung, Analyse und Spezifikation von Anforderungen an ein Software-Produkt befasst.
Diese Anforderungen werden dabei durch die Stakeholder des Software-Produkts definiert.
Insgesamt lassen sich die Anforderungen in drei Kategorien unterteilen: funktionale Anforderungen, Qualitätsanforderungen und Randbedingungen \autocite[vgl.][S. 3]{pohl_basiswissen_2021}.
\newparagraph
Bereits aus den Kategorien der Anforderungen lässt sich der Zusammenhang mit der Softwarequalität ableiten.
Qualitätsanforderungen sind demnach fester Bestandteil des \ac{RE}, da diese die Systemarchitektur maßgeblich beeinflussen können \autocite[vgl.][S. 3-4]{pohl_basiswissen_2021}.
\citeauthor{ebert_systematisches_2019} nennt als Folge eines unzureichenden \ac{RE} eine hohe Zufälligkeit der resultierenden Softwarequalität, da die zu erreichenden Ziele nicht klar definiert sind \autocite[vgl.][S. 51]{ebert_systematisches_2019}. 

Das frühe Einbringen der Qualitätsanforderungen bereits im Prozess des \ac{RE} ist dabei essenziell für die resultierenden Qualitätsmerkmale.
\citeauthor{ebert_systematisches_2019} kommt zu dem Schluss, dass Unternehmen erkannt haben, dass \gqq{Qualität nur dann effektiv und kostengünstig erreichbar ist, wenn sie bereits zu Beginn des Projekts in dessen Zentrum steht - und nicht am Ende \gq{hineingetestet} werden soll} \autocite[S. 66]{ebert_systematisches_2019}.
Durch dieses Vorgehen wird die Anzahl an Änderungen im späteren Projektverlauf reduziert, da die Anforderungen bereits im \ac{RE} definiert wurden und somit nicht mehr verändert werden müssen.

% TODO: @Henry Eventuell RE noch genauer beschreiben? -> Optional