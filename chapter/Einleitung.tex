\section{Grundwissen und Begriffe}
In diesem Kapitel werden die Grundlegenden Begriffe und Konzepte vorgestellt und erläutert.
Dazu wird zunächst der generische Begriff der Qualität definiert und anschließend gegenüber der Software-Qualität abgegrenzt.
Weiterhin wird das Kano-Modell vorgestellt und dessen Bedeutung in Zusammenhang mit dem Requirements Engineering in Bezug auf die Software-Qualität erläutert.
\subsection{Qualität}
Der Begriff Qualität leitet sich von dem lateinischen Begriff \gqq{qualitas} ab, welcher mit den Begriffen \gqq{Beschaffenheit} oder \gqq{Eigenschaft} übersetzt werden kann \autocite[vgl.][]{noauthor_was_nodate}.
Die \ac{ISO} beschreibt den Begriff Qualität als \gqq{Grad, in dem ein Satz inhärenter Merkmale eines Objekts Anforderungen erfüllt} \autocite[S. 17]{iso_iso_2015}.
Die Merkmale, sowie Anforderungen des Objekts werden dabei nicht weiter konkretisiert.
Es handelt sich also um Aspekte, welche nicht universell definiert werden können.
Qualität ist somit ein sehr abstrakter Begriff, welcher je nach Kontext unterschiedlich interpretiert werden kann.

Um eine Aussage über die Qualität eines Objekts tätigen und mit anderen Objekten vergleichen zu können, muss Qualität messbar gemacht werden.
Dafür müssen zunächst die wesentlichen gemeinsamen Merkmale definiert werden.
In einem weiteren Schritt muss dann eine Art und Weise gefunden werden, wie diese spezifischen Merkmale hinsichtlich ihres Umsetzungsgrades bewertet werden können.
Die Bewertung kann dabei sowohl subjektiv als auch objektiv erfolgen.
Das Merkmal Benutzer:innenfreundlichkeit kann beispielsweise durch die Auswertung verschiedener subjektiver Eindrücke von Test-Benutzer:innen bestimmt werden.
Gleichzeitig kann aber auch eine objektive Bewertung vorgenommen werden, indem beispielsweise Faktoren wie die Anzahl an Funktionen zusammen mit der Komplexität der einzelnen Funktionen des Objekts gemessen werden.

\subsection{Software-Qualität}
Der Begriff Software-Qualität stellt eine Spezifizierung des Begriffs Qualität dar, indem der Begriff Qualität in den Kontext Software gestellt wird.
Durch diese Spezifizierung konkretisiert sich auch die Definition des Begriffs (Software-)Qualität, welcher als "Grad, in dem ein Softwareprodukt festgelegte und implizierte Anforderungen erfüllt, wenn es unter bestimmten Bedingungen verwendet wird" \autocite[S. 17]{iso_iso_2011}, definiert werden kann.
Die wesentlichen Merkmale für die Bewertung der Qualität von Software unterteilen sich dabei in zwei Gruppen -- die externe und die interne Qualitätssicht.

Die externe Qualitätssicht bezieht sich auf Software-Merkmale, welche für die anwendende Person die Qualität der Software beeinflusst.
Dazu zählen die vier Merkmale Funktionalität, Laufzeit, Zuverlässigkeit und Benutzbarkeit.

Auf der anderen Seite steht die interne Qualitätssicht, welche sich auf Software-Merkmale, welche für die entwickelnde Person relevant sind, bezieht.
Dazu zählen die Merkmale Wartbarkeit, Transparenz, Übertragbarkeit sowie Testbarkeit \autocite[vgl.][S. 6-10]{hoffmann_software-qualitat_2013}.

\subsection{Kano-Modell}
Damit ein entwickeltes Produkt einen möglichst hohen Gewinn generieren kann, muss das Produkt möglichst oft abgesetzt werden können.
Dies kann unter anderem durch eine möglichst hohe Kundenzufriedenheit erreicht werden.
Um diese systematisch zu erreichen, entwickelte der japanische Professor Noriaki Kano in den 70er Jahren das Kano-Modell \autocite[vgl.][S. 27]{sauerwein_kano-modell_2000}.
Kano definierte dafür fünf Ebenen: Basismerkmale, Leistungsmerkmale, Begeisterungsmerkmale, Unerhebliche-Merkmale und Rückweisungsmerkmale.

\subsection{Requirements Engineering}
d