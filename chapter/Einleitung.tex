\section{Einleitung}
Die Hausarbeit wird im Rahmen der Vorlesung Softwarequalität erstellt.
Dabei werden zunächst Grundwissen und Begriffe definiert, anschließend werden die erlernten Fähigkeiten auf ein konkretes Produkt angewandt.
Hierzu wird das Softwareprodukt \gqq{DEV-CHAT} aus der Vorlesung Software Engineering 1 verwendet.
\newparagraph
%Beschreibung devchat
DEV-CHAT ist ein Chatportal mit mehreren Chatrooms, denen man über einen Schlüssel beitreten kann.
Die Oberfläche ist schlicht im Informatik/Konsolen Stil gehalten und alle Operationen werden über eine Kommandozeile ausgeführt.
Über die Kommandozeile lassen sich Befehle ausführen, um zum Beispiel Direktnachrichten zu senden oder Umfragen zu starten.
\newparagraph
%Aufgabenstellung
Zu Begin der Hausarbeit werden, wie bereits angekündigt, die Begriffe Softwarequalität und Qualität definiert und gegeneinander abgegrenzt, anschließend wird das Kano-Modell erläutert und begründet wieso Requirements Engineering essenziell für eine hohe Softwarequalität ist.
Die Folgeschritte werden auf den DEV-CHAT angewandt, diese gliedern sich in drei Punkte.
Zunächst werden die Anforderungen und Nutzerwartungen definiert, in einem weiteren Schritt werden Fehler, Kosten und Qualitätsmaßnahmen behandelt.
Abschließend wird das Thema Umsetzung und Tests erörtert.