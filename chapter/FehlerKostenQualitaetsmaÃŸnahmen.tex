\section{Fehler, Kosten und Qualitätsmaßnahmen}

% kurze Einleitung zum Thema
In diesem Kapitel wird auf die Fehler, Kosten und Qualitätsmaßnahmen eingegangen.
Hierzu werden typische Faults und Defects des Produkts beschrieben, diese werden durch Fehlerschwere, Priorität und potentielle Fehlerkosten klassifiziert. 
Anschließend werden mehrere Methoden dargestellt, wie die Qualitätskosten in der Entwicklung gesenkt werden können.
Außerdem wird erörtert warum es nicht sinnvoll ist eine vollständige Fehlersicherheit anzustreben.
Abschließend wird für das Produkt eine \ac{FMEA} durchgeführt.

\subsection{Faults und Defects}
% Beschreiben Sie typische Faults and Defects Ihres Software-Produkts. Stellen Sie Fehlerschwere, Priorität und potentielle Fehlerkosten dar.
% Notizen: 
    % Error: Fehlhandlung (Menschliche Handlung die zu einem nicht korrektem Resultat führt Beispiel: Programmierer – Fehler in Programmcode.)
    % Fault: Fehlerzustand (Auftreten eines Fehlers in der Software.)
    % Failure (/Defect): Fehlerwirkung (Bei der Programmausführung kommt es zu einem Abbruch  oder bestimmte Funktionen können nicht wie spezifiziert ausgeführt werden.)

% Aufgaben Aufbau:
    % Faults und Defects [] (-> Tabelle?)
        % Fehlerschwere (1-5) (Prio auch noch mit rein?)
        % Priorität
        % potentielle Fehlerkosten




\subsection{Qualitätskostensenkung und Fehlertoleranz}
%Stellen Sie dar, wie sie die Q-Kosten in der Entwicklung des Produktes senken können? Warum ist es nicht sinnvoll komplette Fehlerfreiheit anzustreben und nennen Sie Methoden und Modelle aus der Vorlesung, die zum Einsatz kommen könnten?
% Wie Entwicklungskosten senken?

% Warum nicht komplette Fehlerfreiheit?

% Methoden aus der Vorlesung um Entwicklungskosten zu senken




\subsection{\acl{FMEA}}

% Failure Mode and Effects Analysis (FMEA) (realistische Annahme für Risikoprioritätszahl RPZ




