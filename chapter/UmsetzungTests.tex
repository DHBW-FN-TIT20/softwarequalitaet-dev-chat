\section{Umsetzung und Tests}
In diesem Kapitel wird auf die Umsetzung und Teststrategie eines Softwareprodukts eingegangen.
Hierzu wird zunächst eine Aufwandsschätzung für die Umsetzung einer Neuentwicklung des \gqq{DEV-CHAT} durchgeführt.
In einem weiteren Schritt wird ein risikobasierter Testplan erstellt und eine Teststrategie entwickelt.
Abschließend werden exemplarisch zwei Testfälle ausformuliert und durchgeführt.

\subsection{Aufwandsschätzung einer Neuentwicklung}
Da es sich bei einem Softwareprodukt nicht um ein herkömmliches materielles Produkt handelt, gestaltet sich die Aufwandsschätzung für die Umsetzung eines Softwareprodukts schwierig.
Besonders der Entwicklungsfortschritt und der Grad der Fertigstellung lassen sich nur schwer messen.
Aus diesem Grund wurden verschiedene Modelle entwickelt, die eine Aufwandsschätzung ermöglichen.
Im Rahmen dieser Arbeit werden das \ac{COCOMO} und die \ac{FPA} genauer betrachtet.

\subsubsection{Constructive Cost Model}
% COCOMO
% 8 KDSI -> PM = 2,4 * KDSI^1,05 -> 2,4 * 8^1,05 = 21,3
% TDEV = 2,5 * PM^0,38 -> 2,5 * 16,8^0,38 = 8
% Personalbedarf: 21,3 / 8 = 2,7
Das \ac{COCOMO} bewertet den Aufwand einer Software anhand der Anzahl der benötigten Programmzeilen.
Im Gegensatz zu dem Entwicklungsaufwand selbst lassen sich die benötigten Programmzeilen leicht abschätzen.
Trotzdem birgt das Verfahren einige Schwächen.
So ist die Abschätzung der benötigten Programmzeilen mit einem hohen Aufwand verbunden und erfordert bereits zu Beginn des Projekts eine technologiespezifische Expertise.
Außerdem werden agile Entwicklungsprozesse nicht berücksichtigt, weshalb sich \ac{COCOMO} besonders in Kombination mit dem Wasserfallmodell anbietet.

\noindent{}Um das \ac{COCOMO} durchführen zu können, muss zunächst die Komplexität des Produkts bestimmt werden.
Diese ergibt sich, wie bereits erwähnt, durch die Anzahl der benötigten Programmzeilen, welche in der Einheit \ac{KDSI} angegeben werden.
Da es sich hier um eine Neuentwicklung handelt, wird die Komplexität der Neuentwicklung in etwa der Komplexität des \gqq{DEV-CHAT} (8~\ac{KDSI}) entsprechen.
Folglich ist das Produkt in die kleinste Komplexitätsstufe, den sogenannten \gqq{Organic Mode}, einzuordnen.

\begin{equation}
  \begin{split}
    \text{\acs{PM}} &= 2,4 \cdot \text{\acs{KDSI}}^{1,05} \\
    \text{\acs{TDEV}} &= 2,5 \cdot \text{\acs{PM}}^{0,38}
  \end{split}
  \label{eq:cocomo}
\end{equation}
\noindent{}Aus diesen Informationen lässt sich nun der Aufwand, gemessen in \ac{PM} ($1~\ac{PM} \equiv$ 19 Arbeitstage zu je 8 Stunden), sowie die \ac{TDEV} berechnen.
Dazu werden die Formeln aus \eqref{eq:cocomo} verwendet.
Es ergibt sich ein Aufwand von ca. 21,3~\ac{PM} ($\approx 3200$ Stunden).
Daraus resultiert eine \ac{TDEV} von ca. 8 Monaten mit einem Personalbedarf von $\frac{21,3}{8} \approx 3$ Personen.

\subsubsection{Function Point Analyse}
% FPA
Bei der \ac{FPA} werden die funktionalen Anforderungen eines Produkts genauer betrachtet und in kleinstmögliche, sinnvolle Aktivitäten zerlegt.
Jedem Elementarprozess wird anschließend ein definierter Punktewert, gemessen in \ac{fp}, zugewiesen.
Die Summe aller Einzelwerte ergibt die Functional-Size des Produkts.

\begin{table}[H]
  \centering
  \begin{tabular}{|l|c|}
    \hline
    \textbf{Elementarprozessart} & \textbf{Punktewert} in \acs{fp} \\
    \hline
    \ac{EI} & \EI \\
    \hline
    \ac{EO} & \EO \\
    \hline
    \ac{EQ} & \EQ \\
    \hline
    \ac{ILF} & \ILF \\
    \hline
    \ac{EIF} & \EIF \\
    \hline
  \end{tabular}
  \caption{Punktewerte nach der Rapid-Näherung}
  \label{tab:rapidNaeherung}
\end{table}
\noindent{}Bei den Elementarprozessen wird zwischen fünf verschiedenen Arten unterschieden.
Tabelle \ref{tab:rapidNaeherung} zeigt die verschiedenen Arten und ihre jeweiligen Punktewerte nach der Rapid-Näherung.

\begin{table}[H]
  \centering
  \begin{tabular}{|l|c|c|}
    \hline
    \textbf{Elementarprozess} & \textbf{Typ} & \textbf{Punktewert} in \acs{fp} \\
    \hline
    Benutzer Datenbank & \acs{ILF} & \ILF \\
    \hline
    Login & \acs{EI} & \EI \\
    \hline
    Registrieren & \acs{EI} & \EI \\
    \hline
    Passwort ändern & \acs{EI} & \EI \\
    \hline
    Chat Datenbank & \acs{ILF} & \ILF \\
    \hline
    Raum erstellen & \acs{EI} & \EI \\
    \hline
    Nachrichten erstellen & \acs{EI} & \EI \\
    \hline
    Nachrichten lesen/synchronisieren & \acs{EQ} & \EQ \\
    \hline
    Umfrage Datenbank & \acs{ILF} & \ILF \\
    \hline
    Kurzbefehle & \acs{EI} & \EI \\
    \hline
    Umfrage erstellen & \acs{EI} & \EI \\
    \hline
    Umfrage ausgeben & \acs{EQ} & \EQ \\
    \hline
    Umfrage abstimmen & \acs{EI} & \EI \\
    \hline
    Rechnung ausführen & \acs{EI} & \EI \\
    \hline
    Ticket Datenbank & \acs{ILF} & \ILF \\
    \hline
    Ticket erstellen & \acs{EI} & \EI \\
    \hline
    Benutzer anzeigen & \acs{EQ} & \EQ \\
    \hline
    Benutzer löschen & \acs{EI} & \EI \\
    \hline
    Benutzer Passwort zurücksetzen & \acs{EI} & \EI \\
    \hline
    Benutzer Rolle vergeben & \acs{EI} & \EI \\
    \hline
    Custom-Chatraum erstellen & \acs{EI} & \EI \\
    \hline
    Chatraum/Umfrage/Tickets löschen & \acs{EI} & \EI \\
    \hline
    Chatraum/Umfrage Ablaufdatum setzen & \acs{EI} & \EI \\
    \hline
    \hline
    \multicolumn{2}{|l|}{\textbf{Summe der Elementarprozessbewertungen}} & \textbf{104} \\ % 16*EI + 3*EQ + 4*ILF
    \hline
  \end{tabular}
  \caption{Funktion Point Analyse der Neuentwicklung}
  \label{tab:FPA}
\end{table}
\noindent{}Basierend auf dieser Bewertungsskala wird die \ac{FPA} für die Neuentwicklung des \gqq{DEV-CHAT} in Tabelle~\ref{tab:FPA} durchgeführt.
Insgesamt ergibt sich eine Punktzahl von 104~\ac{fp}, was einem Zeitaufwand von ca. 1000 Stunden für die Neuentwicklung des \gqq{DEV-CHAT} entspricht.

\subsubsection{Fazit}
% Fazit
% COCOMO: 21,3 PM (3200 Stunden)
% Netzplan: 
% FPA: 1000 Stunden
In den vorangehenden zwei Unterkapiteln wurden zwei verschiedene Methoden zur Schätzung des Aufwands für die Neuentwicklung des \gqq{DEV-CHAT} vorgestellt und durchgeführt.
Die Ergebnisse der zwei Methoden sind in Tabelle \ref{tab:aufwandsschaetzungErgebnisse} zusammengefasst.
\begin{table}[H]
  \centering
  \begin{tabular}{|l|c|}
    \hline
    \textbf{Methoden} & \textbf{Zeitaufwand} in Stunden \\
    \hline
    \ac{COCOMO} & 3200 \\
    \hline
    \ac{FPA} & 1000 \\
    \hline
    \textbf{Durchschnitt} & 2100 \\
    \hline
  \end{tabular}
  \caption{Ergebnisse der Aufwandsschätzung}
  \label{tab:aufwandsschaetzungErgebnisse}
\end{table}

\noindent{}Unter Berücksichtigung beider Schätzungen ergibt sich ein Gesamtaufwand von ungefähr 2100 Stunden für die Neuentwicklung des \gqq{DEV-CHAT}.

\subsection{Testplan und Teststrategie}
% Testplan
% Test-umfang, -kriterien, -methode
Im folgenden Abschnitt wird ein risikobasierter Testplan erstellt und die Teststrategie erläutert.
Hierzu müssen zunächst in einer Risikoanalyse mögliche Risiken identifiziert werden.
Diese wurden bereits im Kapitel~\ref{sec:FMEA} erfasst und mit einer Fehlerschwere und Wahrscheinlichkeit eingeordnet.
\subsubsection{Risikobasierter Testplan}
\subsubsection*{Testumfang}
Um die Qualität des \gqq{DEV-CHAT} sicherzustellen, muss das System getestet werden.
Hierzu werden sowohl Black-Box als auch White-Box Tests verwendet.
Mit Unittests als White-Box Test wird die erwartete Funktionalität des Quellcodes sichergestellt.
Zusätzlich werden Integrationstests verwendet, um das Zusammenspiel der einzelnen Systemkomponenten über die zuvor definierten Schnittstellen zu gewährleisten.
\newparagraph
Als Black-Box Tests werden Systemtests verwendet.
In der Konzeptionsphase wurden die Use-Cases des Systems modelliert und Abläufe definiert.
Mit den Systemtests werden diese Abläufe simuliert und überprüft, ob diese wie erwartet funktionieren.
Abschließend werden Akzeptanz-Tests verwendet um die User-Storys zu verifizieren.
Während dem Betrieb des \gqq{DEV-CHAT} werden Availability-Tests durchgeführt, um die Verfügbarkeit zu gewährleisten.
\subsubsection*{Teststrategie}
Bei der Entwicklung wird der \ac{TDD} Ansatz verfolgt, d.h. dass die Tests vor der Entwicklung der Funktion geschrieben werden.
Zusätzlich wird \ac{CI} verwendet, um einen automatisierten Build der Software zu erstellen und um die Tests auszuführen.
Bei Erstellung der Unittests gilt es zu beachten, dass nicht nur Fehlerfälle oder Grenzfälle, sondern auch normal anzunehmende Werte getestet werden.
Die Integrationstests prüfen primär die korrekte Verbindung von Backend zu Frontend.
Hierzu wird das System in einer Demo-Umgebung ausgeführt.
Innerhalb dieser Umgebung werden auch die Systemtests als Black-Box Test ausgeführt.
\newparagraph
Die Unittests, Integrationstests und Systemtests werden automatisiert in einer Pipeline im Rahmen von \ac{CI} ausgeführt und dokumentiert.
Die Akzeptanz-Tests zur Verifikation der User-Storys werden manuell am Produkt auf einem Test-Server ausgeführt und manuell dokumentiert. 
Sind alle Tests bestanden kann ein Release freigegeben werden.
Läuft das System in der Production werden zyklisch Availability-Tests durchgeführt, um den Zugang zu \gqq{DEV-CHAT} und der Datenbank sicherzustellen.
\subsection{Exemplarische Durchführung}
\subsubsection{Testfall 1: Benutzer registrieren}
% Testfall 1: Benutzer registrieren
Bei diesem Test wird überprüft, ob ein Benutzer sich registrieren kann.
\paragraph{Ablaufbeschreibung} \mbox{}\\
Ausgangssituation: Instanz wird ausgeführt und ist mit dem Browser erreichbar. Der Benutzer ist noch nicht registriert. Folgende Schritte:

\begin{enumerate}
  \item Website Aufrufen.
  \item Registrierungsseite aufrufen.
  \item Benötigte Registrierungsdaten eingeben, z.B. Name, Passwort, Passwort wiederholen.
  \item Verifizieren, ob der Benutzer erfolgreich registriert wurde und auf die Startseite weitergeleitet wurde.
\end{enumerate}

\paragraph{Durchführung}
\begin{enumerate}
  \item Browser geöffnet und Website über \gqq{http://localhost:3001} aufgerufen. \newline
  → Website erscheint nach kurzer Ladezeit.
  \item Auf den Button \gqq{Create Account} geklickt. \newline
  → Registrierungsseite wird mit einem leeren Formular angezeigt.
  \item Registrierungsdaten eingeben: Name, Passwort, Passwort wiederholen. \newline
  → Registrierungsdaten werden im Formular angezeigt.
  \item Auf \gqq{Create} geklickt. \newline
  → Benutzer wird registriert, dessen Daten auf die Datenbank gespeichert und auf die Startseite weitergeleitet.
\end{enumerate}

\subsubsection{Testfall 2: Nachrichten senden}
% Testfall 2: Nachrichten senden
Bei diesem Test wird überprüft, ob ein Benutzer Nachrichten senden kann.
\paragraph{Ablaufbeschreibung} \mbox{}\\
Ausgangssituation: Instanz wird ausgeführt und ist mit dem Browser erreichbar. Der Benutzer ist eingeloggt. Folgende Schritte: 

\begin{enumerate}
  \item Website Aufrufen.
  \item Bestehenden Chatraum mit einem Chat-Key beitreten oder neuen Chatraum erstellen.
  \item Nachricht eingeben.
  \item Nachricht senden.
  \item Verifizieren, ob die Nachricht erfolgreich gesendet wurde.
\end{enumerate}

\paragraph{Durchführung}
\begin{enumerate}
  \item Browser geöffnet und Website über \gqq{http://localhost:3001} aufgerufen. \newline
  → Website erscheint nach kurzer Ladezeit.
  \item In das Textfeld \gqq{Chat-Key} einen bestehenden Key eingeben oder/und auf den Button \gqq{Join} klicken. \newline
  → Der Benutzer wird in einen neuen oder bestehenden Chatraum weitergeleitet.
  \item Nachricht eingeben. \newline
  → Nachricht in dem Textfeld \gqq{Gib eine Nachricht ein} eingeben.
  \item Mit \gqq{Enter} die Nachricht senden. \newline
  → Nachricht wird gesendet und in dem Chatraum angezeigt.
\end{enumerate}

\subsubsection{Testfall 3: Administrationsbereich aufrufen}
% Testfall 3: Administrationsbereich aufrufen
Bei diesem Test wird überprüft, ob ein Benutzer den Administrationsbereich aufrufen kann.
\paragraph{Ablaufbeschreibung} \mbox{}\\
Ausgangssituation: Instanz wird ausgeführt und ist mit dem Browser erreichbar. Der Benutzer ist eingeloggt. Folgende Schritte: 
\begin{enumerate}
  \item Website Aufrufen.
  \item Den Button \gqq{Admin Settings} klicken.
  \item Verifizieren, ob der Benutzer auf die Administrationsseite weitergeleitet wurde.
\end{enumerate}

\paragraph{Durchführung}
\begin{enumerate}
  \item Browser geöffnet und Website über \gqq{http://localhost:3001} aufgerufen. \newline
  → Website erscheint nach kurzer Ladezeit.
  \item Auf den Button \gqq{Admin Settings} geklickt. \newline
  → Der Benutzer wird auf die Administrationsseite weitergeleitet.
\end{enumerate}