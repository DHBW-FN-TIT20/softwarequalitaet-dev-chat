\section{Umsetzung und Tests}
In diesem Kapitel wird auf die Umsetzung und Teststrategie eines Software Produkts eingegangen.
Hierzu wird zunächst eine Aufwandsschätzung für die Umsetzung einer Neuentwicklung des \gqq{DEV-CHAT} durchgeführt.
In einem weiteren Schritt wird ein risikobasierter Testplan erstellt und eine Teststrategie entwickelt.
Abschließend werden exemplarisch zwei Testfälle ausformuliert und durchgeführt.

\subsection{Aufwandsschätzung einer Neuentwicklung}
Da es sich bei einem Software Produkt nicht um ein herkömmliches materielles Produkt handelt, gestaltet sich die Aufwandsschätzung für die Umsetzung eines Software Produkts schwierig.
Besonders der Entwicklungsfortschritt und der Grad der Fertigstellung lassen sich nur schwer messen.
Aus diesem Grund wurden verschiedene Modelle entwickelt, die eine Aufwandsschätzung ermöglichen.
Im Rahmen dieser Arbeit werden das \ac{COCOMO}, der Netzplan und die \ac{FPA} genauer betrachtet.

\subsubsection{Constructive Cost Model}
% COCOMO
% 8 KDSI -> PM = 2,4 * KDSI^1,05 -> 2,4 * 8^1,05 = 21,3
% TDEV = 2,5 * PM^0,38 -> 2,5 * 16,8^0,38 = 8
% Personalbedarf: 21,3 / 8 = 2,7
Das \ac{COCOMO} bewertet den Aufwand einer Software anhand der Anzahl der benötigten Programmzeilen.
Im Gegensatz zu dem Entwicklungsaufwand selbst, lassen sich die benötigten Programmzeilen leicht abschätzen.
Trotzdem birgt das Verfahren einige Schwächen.
So ist die Abschätzung der benötigten Programmzeilen mit einem hohen Aufwand verbunden und erfordert bereits zu Beginn des Projekts eine technologiespezifische Expertise.
Außerdem werden agile Entwicklungsprozesse nicht berücksichtigt, weshalb sich \ac{COCOMO} besonders in Kombination mit dem Wasserfallmodell anbietet.

\noindent{}Um das \ac{COCOMO} durchführen zu können, muss zunächst die Komplexität des Produkts bestimmt werden.
Diese ergibt sich, wie bereits erwähnt, durch die Anzahl der benötigten Programmzeilen, welche in der Einheit \ac{KDSI} angegeben werden.
Da es sich hier um eine Neuentwicklung handelt, wird die Komplexität der Neuentwicklung in etwa der Komplexität des \gqq{DEV-CHAT} (8~\ac{KDSI}) entsprechen.
Folglich ist das Produkt in die kleinste Komplexitätsstufe, den sogenannten \gqq{Organic Mode}, einzuordnen.

\begin{equation}
  \begin{split}
    \text{\acs{PM}} &= 2,4 \cdot \text{\acs{KDSI}}^{1,05} \\
    \text{\acs{TDEV}} &= 2,5 \cdot \text{\acs{PM}}^{0,38}
  \end{split}
  \label{eq:cocomo}
\end{equation}
\noindent{}Aus diesen Informationen lässt sich nun der Aufwand, gemessen in \ac{PM} ($1~\ac{PM} \equiv$ 19 Arbeitstage zu je 8 Stunden), sowie die \ac{TDEV} berechnen.
Dazu werden die Formeln aus \eqref{eq:cocomo} verwendet.
Es ergibt sich ein Aufwand von ca. 21,3~\ac{PM}.
Daraus resultiert eine \ac{TDEV} von ca. 8 Monaten mit einem Personalbedarf von $\frac{21,3}{8} = 2,7$ Personen.

\subsubsection{Netzplan}
% Netzplan

\subsubsection{Function Point Analyse}
% FPA

\subsection{Testplan und Teststrategie}


\subsection{Exemplarische Durchführung}