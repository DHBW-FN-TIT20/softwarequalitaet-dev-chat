\section{Umsetzung und Tests}
In diesem Kapitel wird auf die Umsetzung und Teststrategie eines Software Produkts eingegangen.
Hierzu wird zunächst eine Aufwandsschätzung für die Umsetzung einer Neuentwicklung des \gqq{DEV-CHAT} durchgeführt.
In einem weiteren Schritt wird ein risikobasierter Testplan erstellt und eine Teststrategie entwickelt.
Abschließend werden exemplarisch zwei Testfälle ausformuliert und durchgeführt.

\subsection{Aufwandsschätzung einer Neuentwicklung}
Da es sich bei einem Software Produkt nicht um ein herkömmliches materielles Produkt handelt, gestaltet sich die Aufwandsschätzung für die Umsetzung eines Software Produkts schwierig.
Besonders der Entwicklungsfortschritt und der Grad der Fertigstellung lassen sich nur schwer messen.
Aus diesem Grund wurden verschiedene Modelle entwickelt, die eine Aufwandsschätzung ermöglichen.
Im Rahmen dieser Arbeit werden das \ac{COCOMO}, der Netzplan und die \ac{FPA} genauer betrachtet.

\subsubsection{Constructive Cost Model}
% COCOMO
% 8 KDSI -> PM = 2,4 * KDSI^1,05 -> 2,4 * 8^1,05 = 21,3
% TDEV = 2,5 * PM^0,38 -> 2,5 * 16,8^0,38 = 8
% Personalbedarf: 21,3 / 8 = 2,7
Das \ac{COCOMO} bewertet den Aufwand einer Software anhand der Anzahl der benötigten Programmzeilen.
Im Gegensatz zu dem Entwicklungsaufwand selbst, lassen sich die benötigten Programmzeilen leicht abschätzen.
Trotzdem birgt das Verfahren einige Schwächen.
So ist die Abschätzung der benötigten Programmzeilen mit einem hohen Aufwand verbunden und erfordert bereits zu Beginn des Projekts eine technologiespezifische Expertise.
Außerdem werden agile Entwicklungsprozesse nicht berücksichtigt, weshalb sich \ac{COCOMO} besonders in Kombination mit dem Wasserfallmodell anbietet.

\noindent{}Um das \ac{COCOMO} durchführen zu können, muss zunächst die Komplexität des Produkts bestimmt werden.
Diese ergibt sich, wie bereits erwähnt, durch die Anzahl der benötigten Programmzeilen, welche in der Einheit \ac{KDSI} angegeben werden.
Da es sich hier um eine Neuentwicklung handelt, wird die Komplexität der Neuentwicklung in etwa der Komplexität des \gqq{DEV-CHAT} (8~\ac{KDSI}) entsprechen.
Folglich ist das Produkt in die kleinste Komplexitätsstufe, den sogenannten \gqq{Organic Mode}, einzuordnen.

\begin{equation}
  \begin{split}
    \text{\acs{PM}} &= 2,4 \cdot \text{\acs{KDSI}}^{1,05} \\
    \text{\acs{TDEV}} &= 2,5 \cdot \text{\acs{PM}}^{0,38}
  \end{split}
  \label{eq:cocomo}
\end{equation}
\noindent{}Aus diesen Informationen lässt sich nun der Aufwand, gemessen in \ac{PM} ($1~\ac{PM} \equiv$ 19 Arbeitstage zu je 8 Stunden), sowie die \ac{TDEV} berechnen.
Dazu werden die Formeln aus \eqref{eq:cocomo} verwendet.
Es ergibt sich ein Aufwand von ca. 21,3~\ac{PM} ($\approx 3200$ Stunden).
Daraus resultiert eine \ac{TDEV} von ca. 8 Monaten mit einem Personalbedarf von $\frac{21,3}{8} \approx 3$ Personen.

\subsubsection{Netzplan}
% Netzplan

\subsubsection{Function Point Analyse}
% FPA
Bei der \ac{FPA} werden die Funktionalen Anforderungen eines Produkts genauer betrachtet und in kleinstmögliche, sinnvolle Aktivitäten zerlegt.
Jedem Elementarprozess wird anschließend ein definierter Punktewert, gemessen in \ac{fp}, zugewiesen.
Die Summe aller Einzelwerte ergibt die Functional-Size des Produkts.

\begin{table}
  \centering
  \begin{tabular}{|l|c|}
    \hline
    \textbf{Elementarprozessart} & \textbf{Punktewert} in \acs{fp} \\
    \hline
    \ac{EI} & \EI \\
    \hline
    \ac{EO} & \EO \\
    \hline
    \ac{EQ} & \EQ \\
    \hline
    \ac{ILF} & \ILF \\
    \hline
    \ac{EIF} & \EIF \\
    \hline
  \end{tabular}
  \caption{Punktewerte nach der Rapid-Näherung}
  \label{tab:rapidNaeherung}
\end{table}
\noindent{}Bei den Elementarprozessen wird zwischen fünf verschiedenen Arten unterschieden.
Tabelle \ref{tab:rapidNaeherung} zeigt die verschiedenen Arten und ihre jeweiligen Punktewerte nach der Rapid-Näherung.

\begin{table}
  \centering
  \begin{tabular}{|l|c|c|}
    \hline
    \textbf{Elementarprozess} & \textbf{Typ} & \textbf{Punktewert} in \acs{fp} \\
    \hline
    Benutzer Datenbank & \acs{ILF} & \ILF \\
    \hline
    Login & \acs{EI} & \EI \\
    \hline
    Registrieren & \acs{EI} & \EI \\
    \hline
    Passwort ändern & \acs{EI} & \EI \\
    \hline
    Chat Datenbank & \acs{ILF} & \ILF \\
    \hline
    Raum erstellen & \acs{EI} & \EI \\
    \hline
    Nachrichten erstellen & \acs{EI} & \EI \\
    \hline
    Nachrichten lesen/synchronisieren & \acs{EQ} & \EQ \\
    \hline
    Umfrage Datenbank & \acs{ILF} & \ILF \\
    \hline
    Kurzbefehle & \acs{EI} & \EI \\
    \hline
    Umfrage erstellen & \acs{EI} & \EI \\
    \hline
    Umfrage ausgeben & \acs{EQ} & \EQ \\
    \hline
    Umfrage abstimmen & \acs{EI} & \EI \\
    \hline
    Rechnung ausführen & \acs{EI} & \EI \\
    \hline
    Ticket Datenbank & \acs{ILF} & \ILF \\
    \hline
    Ticket erstellen & \acs{EI} & \EI \\
    \hline
    Benutzer anzeigen & \acs{EQ} & \EQ \\
    \hline
    Benutzer löschen & \acs{EI} & \EI \\
    \hline
    Benutzer Passwort zurücksetzen & \acs{EI} & \EI \\
    \hline
    Benutzer Rolle vergeben & \acs{EI} & \EI \\
    \hline
    Custom-Chat-Raum erstellen & \acs{EI} & \EI \\
    \hline
    Chat-Raum/Umfrage/Tickets löschen & \acs{EI} & \EI \\
    \hline
    Chat-Raum/Umfrage Ablaufdatum setzen & \acs{EI} & \EI \\
    \hline
    \hline
    \multicolumn{2}{|l|}{\textbf{Summe der Elementarprozessbewertungen}} & \textbf{104} \\ % 16*EI + 3*EQ + 4*ILF
    \hline
  \end{tabular}
  \caption{Funktion Point Analyse der Neuentwicklung}
  \label{tab:FPA}
\end{table}
\noindent{}Basierend auf dieser Bewertungsskala wird die \ac{FPA} für die Neuentwicklung des \gqq{DEV-CHAT} in Tabelle~\ref{tab:FPA} durchgeführt.
Insgesamt ergibt sich eine Punktzahl von 104~\ac{fp}, was einem Zeitaufwand von ca. 1000 Stunden für die Neuentwicklung des \gqq{DEV-CHAT} entspricht.

\subsubsection{Fazit}
% Fazit
% COCOMO: 21,3 PM (3200 Stunden)
% Netzplan: 
% FPA: 1000 Stunden

\subsection{Testplan und Teststrategie}


\subsection{Exemplarische Durchführung}