\section{Fehler, Kosten und Qualitätsmaßnahmen}

% kurze Einleitung zum Thema




\subsection{Beschreiben Sie typische Faults and Defects Ihres Software-Produkts. Stellen Sie Fehlerschwere, Priorität und potentielle Fehlerkosten dar.}

% Notizen: 
    % Error: Fehlhandlung (Menschliche Handlung die zu einem nicht korrektem Resultat führt Beispiel: Programmierer – Fehler in Programmcode.)
    % Fault: Fehlerzustand (Auftreten eines Fehlers in der Software.)
    % Failure (/Defect): Fehlerwirkung (Bei der Programmausführung kommt es zu einem Abbruch  oder bestimmte Funktionen können nicht wie spezifiziert ausgeführt werden.)

% Aufgaben Aufbau:
    % Faults und Defects [] (-> Tabelle?)
        % Fehlerschwere (1-5) (Prio auch noch mit rein?)
        % Priorität
        % potentielle Fehlerkosten




\subsection{Stellen Sie dar, wie sie die Q-Kosten in der Entwicklung des Produktes senken können? Warum ist es nicht sinnvoll komplette Fehlerfreiheit anzustreben und nennen Sie Methoden und Modelle aus der Vorlesung, die zum Einsatz kommen könnten?}

% Wie Entwicklungskosten senken?

% Warum nicht komplette Fehlerfreiheit?

% Methoden aus der Vorlesung um Entwicklungskosten zu senken




\subsection{Führen Sie eine FMEA für Ihr Produkt durch – für die RPZ treffen Sie realistische Annahmen}

% Failure Mode and Effects Analysis (FMEA) (realistische Annahme für Risikoprioritätszahl RPZ




